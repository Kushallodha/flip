\documentclass{article}
\usepackage{graphicx} % Required for inserting images
\usepackage{amsmath}
\title{rcrk24}
\author{Alex Kim}
\date{July 2024}

\begin{document}

 \maketitle

\section{Model Description}
The {\tt rcrk24} model includes redshift dependence and uses the first-order approximation for the evolution of growth
\begin{equation}
{P_{xy}} = {\sigma^2_8} \frac{P^{(f)}_{xy}}{\sigma^{2(f)}_8},
\end{equation}
where the ${}^{(f)}$ superscript refers to the input fiducial model.

For observables $\alpha$, $\beta$ at $z_1$, $z_2$.
\begin{equation}
P_{\alpha\beta}(z_1,z_2)= A_\alpha(z_1) A_\beta(z_2) \frac{P^{(f)}_{xy}}{\sigma^{2(f)}_8}
\end{equation}
The implementation in {\tt flip} is based on specifying $A_\alpha$ in {\tt coefficients.get\_coefficients} and
its partial derivatives in {\tt fisher\_terms.get\_partial\_derivative\_coefficients}.

\section{Variants}

There are two variants:
\begin{description}
\item[growth\_rate] Constant $f\sigma_8$.
\item[growth\_index] $f=\Omega_{M}^\gamma$ with free parameters $\Omega_{M0}$ and $\gamma$.
\end{description}
In both variants, redshift dependence enters in the scale factor $a$ the Hubble parameter $H$.
In the {\tt growth\_index} variant $\Omega_M$ is redshift dependent.

\subsection{growth\_rate}

The parameters that describe the model are 
\begin{description}
\item[fs8] $f\sigma_8$
\end{description}

\subsubsection{$A_v$}
\begin{align}
A_{v}(z) & = (1+z)^{-1} H(z) f\sigma_8 \\
\frac{\partial A_{v}}{\partial f\sigma_8} (z) & = (1+z)^{-1} H(z)
\end{align}

\subsection{growth\_index}
The following functions are associated with the model
\begin{align}
\sigma_8(a) & = \sigma_{8,\text{cmb}}  e^{\int^{\ln{a}}_{\ln{a_\text{CMB}}} \Omega^\gamma d\ln{a'}} \\
\frac{d\Omega}{d\Omega_{M0}} & = \frac{a^{-3}}{\Omega_{M0} a^{-3} + 1 - \Omega_{M0}} - \frac{\Omega_{M0} a^{-3}(a^{-3}-1)}{(\Omega_{M0} a^{-3} + 1 - \Omega_{M0})^2}.
\end{align}
They are implemented in {\tt flip\_terms.py}.

The value of $ \sigma_{8,\text{cmb}}$ is currently not hardwired but is left as a user-specified parameter.
  A reasonable choice is $ \sigma_{8,\text{cmb}} = 0.832 * 0.001176774706956903$.

The parameters that describe the model are 
\begin{description}
\item[gamma] $\gamma$
\item[Om0] $\Omega_{M0}$
\item[s8\_cmb]  $\sigma_{8,\text{cmb}}$
\end{description}

\subsubsection{$A_v$}

\begin{align}
A_{v}(z) & = (1+z)^{-1} H(z) \Omega^\gamma  \sigma_8(a) \\
\frac{\partial A_{v}}{\partial \gamma} (z) & = (1+z)^{-1} H(z)
\left( \ln{\Omega}   + 
 e^{\int^{\ln{a}}_{\ln{a_\text{CMB}}}\left( \ln{\Omega} \right) \Omega^\gamma d\ln{a}}
\right) \Omega^\gamma  \sigma_8(a) \\
\frac{\partial A_{v}}{\partial \Omega_0} (z) & = (1+z)^{-1} H(z) 
\left( \gamma  \frac{d\Omega}{d\Omega_{M0}} \Omega^{-1}  + 
 e^{\int^{\ln{a}}_{\ln{a_\text{CMB}}}\gamma \frac{d\Omega}{d\Omega_{M0}} \Omega^{\gamma-1} d\ln{a'}}
\right) \Omega^\gamma  \sigma_8(a).
\end{align}
$A_v$ is implemented in {\tt coefficients.get_coefficients}.
The partial derivatives are implemented in both {\tt fisher_terms.get\_partial\_derivative\_coefficeints} and {\tt flip\_terms}.

For numerical efficiency it is useful to use
\begin{align}
\sigma_8(a) & =\sigma_{8,\text{cmb}}  
 e^{\int^1_{\ln{a_\text{CMB}}} \Omega^\gamma d\ln{a} - \int_{\ln{a}}^1 \Omega^\gamma d\ln{a'} } 
\end{align}
and
\begin{equation}
\int_{\ln{a}}^1 \Omega_M^\gamma d\ln{a'} \approx -\ln{a} \left[\Omega_{M0}^\gamma + \Omega_{M0}^\gamma 3\gamma(1-\Omega_{M0} )\right] 
- (1-a) \Omega_{M0}^\gamma 3\gamma (1-\Omega_{M0}) d\ln{x}.
\end{equation}
to approximate $A$ and its partial derivatives.  These are implemented in {\tt flip\_terms}, where there is a hardwired variable {\tt exact=False} that
makes the code run using approximations by default.

flip has a fiducial power spectrum $P^F_{xy}$ at $z=0$
with $\sigma_8^F$,
where $x,y \in \{g, \theta \}$.  The convention
is to normalize this so that what is really available is
$\mathcal{P}^F_{xy} = \sigma_8^{F\,-2} P^F_{xy}$.

In linear theory the
power spectrum is proportional to $\sigma_8$ as a function of redshift, and the $\alpha \beta$ power
spectrum is some factor times the $xy$ power spectrum, so
\begin{align}
    P_{\alpha,\beta}(z_1,z_2;p)
    &=
    \frac{\sigma_8(z_1; p)}{\sigma_{8}(0; p)}
    \frac{\sigma_8(z_2; p)}{\sigma_{8}(0; p)}
    A_{\alpha,\beta}(z_1,z_2;p) P_{x,y}(z_1=0, z_2=0;p) \\    
 & =     \frac{\sigma_8(z_1; p)}{\sigma_{8}(0; p)}
    \frac{\sigma_8(z_2; p)}{\sigma_{8}(0; p)}
    A_{\alpha,\beta}(z_1,z_2;p)
    \frac{P_{xy}(z_1=0, z_2=0;p)}
 {P^F_{xy}} P^F_{xy} \\
  & = \left(\sigma_8^{F}\right)^2 \frac{\sigma_8(z_1; p)}{\sigma_{8}(0; p)}
    \frac{\sigma_8(z_2; p)}{\sigma_{8}(0; p)}
    A_{\alpha,\beta}(z_1,z_2;p)
    \frac{P_{xy}(z_1=0, z_2=0;p)}
 {P^F_{xy}}  \mathcal{P}^F_{xy}.
\end{align}
For example $A_{vv}=(aHf)(z_1)(aHf)(z_2)$.

In flip the power spectrum is constructed as
\begin{align}
    P_{\alpha,\beta}(z_1,z_2;p) = A_{\alpha,\beta}(z_1,z_2, p) B_{\alpha,\beta}(z) \mathcal{P}^F_{xy},
\end{align}
where $B_{\alpha,\beta}$ is defined by the method {\tt power\_spectrum\_amplitude\_function} define in
{\tt flip\_terms.py} and
the combination $A_{\alpha,\beta}(p) B_{\alpha,\beta}(z)$ is given by the variable {\tt coefficient\_vector}
defined in the file {\tt coefficient.py}.
Given this convention,
\begin{equation}
    B_{\alpha,\beta}(z;p) = \left(\sigma_8^{F}\right)^2 \frac{\sigma_8(z_1; p)}{\sigma_{8}(0; p)}
    \frac{\sigma_8(z_2; p)}{\sigma_{8}(0; p)}
    \frac{P_{xy}(z_1=0, z_2=0;p)}
 {P^F_{xy}}. 
\end{equation}
The dependence of $B$ on $p$, is important to recognized: $B$ contributes to terms in the Fisher Matrix calculation and the current flip API does not have
$B$ as a function of $p$.

For now flip assumes that $P^F_{xy}$ is at $z=0$ and the shape of the power spectrum is model independent.  Then
\begin{equation}
    \frac{P_{xy}(z_1=0, z_2=0;p)} 
 {P^F_{xy}} = \left(\frac{\sigma_8(0; p)}{\sigma^F_{8}} \right)^2
\end{equation}
and
\begin{equation}
    B_{\alpha,\beta}(z;p) = \sigma_8(z_1; p)
    \sigma_8(z_2; p). 
\end{equation}


By definition
$\sigma_8(z) = \sigma_{8,0} \exp{\int_1^a f(x;p) d\ln(x)}$ so
\begin{equation}
    \frac{d \sigma_8}{dp} = \sigma_8 \frac{d
    \left(\int_1^a f d\ln(x)\right)} {dp}
\end{equation}

For the case where $f=\Omega_M^\gamma$
\begin{equation}
 \int^a_1 \frac{dx}{x}\,\Omega(x)^\gamma \approx \ln a \,\left[\Omega_0^\gamma + \Omega_0^\gamma 3\gamma(1-\Omega_0)\right] + (1-a) \Omega_0^\gamma 3\gamma(1-\Omega_0)
\end{equation}

For the case where $f\sigma_8=c$
is assumed to be constant
\begin{align}
\sigma_8 & = c \left( c^{-1}\sigma_{8,0} + \ln{a}\right) \\
f & = \frac{d\ln{\sigma_8}}{d\ln{a}}  = \frac{1}{  c^{-1}\sigma_{8,0} + \ln{a}}
\end{align}


\paragraph{Tentative correction}
\


Without normalizing power spectrum, we have from theory

\begin{equation}
       P_{vv}(z_1,z_2;p) = (aHf)(z_1)(aHf)(z_2) P_{\theta\theta}(z_1, z_2;p) 
\end{equation}

For the models we developed until now, we were not accounting for redshift dependency (only at $z=0$). We normalize the power spectrum $P_{\theta\theta}$ by $\sigma_8^F$ so that the parameter to fit is not $f$ but $f\sigma_8$. It means that for this model we have:

\begin{equation}
       P_{vv}(z;p) = (H_0)^2 (f\sigma_8)^2 \frac{P_{\theta\theta}(z=0;p)}{\sigma_8^{F, 2}} = (H_0)^2 (f\sigma_8)^2  \mathcal{P}^F_{\theta\theta}(z=0;p)
\end{equation}

\noindent taking the previous notation $\mathcal{P}^F_{xy} = \sigma_8^{F\,-2} P^F_{xy}$. The only parameter fitted here is $f\sigma_8$ at redshift $z=0$. There is one assumption on those models: we are not considering the power spectrum dependency on other parameters $p$, i.e. we are considering that 
$\mathcal{P}_{\theta\theta}(z=0;p) = \mathcal{P}^F_{\theta\theta}(z=0)$. This can be easily extended to fitting at a redshift $z \neq 0$.


Now let's develop a full model for the redshift dependency. We want to measure $f\sigma_8$ at redshift $z=0$ for objects located at different redshifts ($z_1$, $z_2$). 

\begin{align}
    P_{vv}(z_1,z_2;p)
    &= (aHf)(z_1)(aHf)(z_2) P_{\theta\theta}(z_1, z_2;p) \\    
 & = (aHf)(z_1)(aHf)(z_2)  \frac{\sigma_8(z_1; p)}{\sigma_{8}(0; p)} \frac{\sigma_8(z_2; p)}{\sigma_{8}(0; p)} P_{\theta\theta}(z=0;p) \\
 & = (aHf)(z_1)(aHf)(z_2)  \frac{\sigma_8(z_1; p)}{\sigma_{8}(0; p)} \frac{\sigma_8(z_2; p)}{\sigma_{8}(0; p)} \frac{P_{\theta\theta}(z=0;p)}{P^{F}_{\theta\theta}(z=0)} P^{F}_{\theta\theta}(z=0) \\
 & = (aHf\sigma_8)(z_1)(aHf\sigma_8)(z_2) \frac{P_{\theta\theta}(z=0;p)}{\sigma_{8}(0; p)^2 P^{F}_{\theta\theta}(z=0)} P^{F}_{\theta\theta}(z=0) \\
 & = (aHf\sigma_8)(z_1)(aHf\sigma_8)(z_2) \frac{\sigma_{8}^{F\,2} }{\sigma_{8}(0; p)^2 } \frac{P_{\theta\theta}(z=0;p)}{P^{F}_{\theta\theta}(z=0)} \mathcal{P}^{F}_{\theta\theta}(z=0) 
\end{align}




\begin{align}
    P_{vv}(z_1,z_2;p)
    &= (aHf)(z_1)(aHf)(z_2) P_{\theta\theta}(z_1, z_2;p) \\    
 & = (aH)(z_1) (aH)(z_2) \frac{f(z_1)f(z_2)}{f(0)f(0)}  f(0)f(0) \frac{\sigma_8(z_1; p)}{\sigma_{8}(0; p)} \frac{\sigma_8(z_2; p)}{\sigma_{8}(0; p)} P_{\theta\theta}(z=0;p) \\
 & = (aH)(z_1) (aH)(z_2)  (f\sigma_8)^2 \frac{f\sigma_8(z_1; p)}{f\sigma_{8}(0; p)} \frac{f\sigma_8(z_2; p)}{f\sigma_{8}(0; p)} \mathcal{P}_{\theta\theta}(z=0;p) \\ 
 & = (aH)(z_1) (aH)(z_2)  (f\sigma_8)^2 \frac{f\sigma_8(z_1; p)}{f\sigma_{8}(0; p)} \frac{f\sigma_8(z_2; p)}{f\sigma_{8}(0; p)} \frac{P_{\theta\theta}(z=0;p)}{P^{F}_{\theta\theta}(z=0)} \mathcal{P}^{F}_{\theta\theta}(z=0) \\ 
\end{align}


Here we kept a $f\sigma_8$ modeling (i.e. no growth index) and we are considering the impact of $p$ on the power spectrum. The power spectrum amplitude term is:

\begin{equation}
    B_{vv}(p) = (aH)(z_1) (aH)(z_2)  \frac{f\sigma_8(z_1; p)}{f\sigma_{8}(0; p)} \frac{f\sigma_8(z_2; p)}{f\sigma_{8}(0; p)} \frac{P_{\theta\theta}(z=0;p)}{P^{F}_{\theta\theta}(z=0)}
\end{equation}

Going back to the previous flip models, with $z_1 = z_2 = 0$, and the dependence on $p$ of the power spectrum unaccounted, we get the power\_spectrum\_amplitude\_function $B_{vv} = H_0^2$, and coefficients\_dict['vv'] = $f \sigma_8^F$. We accounted for the $B_{vv}$ term by multiplying the covariance by $100^2$, and expressing all coordinates in Mpc.$h^{-1}$. 


We want to go for a growth index modeling, which is more appropriate for a redshift dependency, as it will make clearer some dependency on $p$. 

\begin{align}
    P_{vv}(z_1,z_2;p)
    &= (aHf)(z_1)(aHf)(z_2) P_{\theta\theta}(z_1, z_2;p) \\ 
    &= (aH)(z_1)(aH)(z_2) \Omega_{m}(z_1)^\gamma \Omega_{m}(z_2)^\gamma P_{\theta\theta}(z_1, z_2;p) \\  
    &= (aH)(z_1)(aH)(z_2) \Omega_{m}(0)^{2\gamma} \left(\frac{\Omega_{m}(z_1) \Omega_{m}(z_2)}{\Omega_{m}(0)^{2}}\right)^\gamma \frac{\sigma_8(z_1; p)}{\sigma_{8}(0; p)} \frac{\sigma_8(z_2; p)}{\sigma_{8}(0; p)} P_{\theta\theta}(z=0;p) \\  
 & = (aH)(z_1)(aH)(z_2) \Omega_{m}(0)^{2\gamma} \sigma_8^2 \left(\frac{\Omega_{m}(z_1) \Omega_{m}(z_2)}{\Omega_{m}(0)^{2}}\right)^\gamma   \\ 
 &  \frac{P_{\theta\theta}(z=0;p)}{P^{F}_{\theta\theta}(z=0)} \frac{\sigma_8(z_1; p)}{\sigma_{8}(0; p)} \frac{\sigma_8(z_2; p)}{\sigma_{8}(0; p)} \mathcal{P}^{F}_{\theta\theta}(z=0) \\ 
\end{align}

And we obtain:

\begin{equation}
    B_{vv}(p) = (aH)(z_1)(aH)(z_2)\left(\frac{\Omega_{m}(z_1) \Omega_{m}(z_2)}{\Omega_{m}(0)^{2}}\right)^\gamma  \frac{P_{\theta\theta}(z=0;p)}{P^{F}_{\theta\theta}(z=0)} \frac{\sigma_8(z_1; p)}{\sigma_{8}(0; p)} \frac{\sigma_8(z_2; p)}{\sigma_{8}(0; p)} 
\end{equation}

Here, if we are considering that the variation of power spectra on $p$ only depends on $\Omega_{m}$ (i.e. considering flat lcdm for this second order effect), the three parameters to fit are $\Omega_{m}$, $\gamma$, and $\sigma_8$.



Summary
\begin{itemize}
    \item $\text{coefficients\_dict['vv']} = aHf$
    \item $\text{power\_spectrum\_amplitude\_function} = \sigma_8$ (current flip implementation) or $\sigma_8/\sigma_{8, fid}$ (candidate new implementation)
\end{itemize}
\end{document}
